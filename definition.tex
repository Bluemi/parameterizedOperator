\documentclass{article}
\usepackage{fullpage}
\usepackage{amsmath}
\usepackage{amssymb}

\begin{document}
\section{Parameterized Operator}
\paragraph{For - Definition}

\[ \operatorname*{cfor}_{x=a}^{b}\left(\$\ast c,n\right) = \operatorname*{for}_{x=a}^{b}\left(c,\ast \right) \]
\[ \operatorname*{for}_{x=1}^{b}\left(c, \ast\right) = \operatorname*{for}^{b}\left(c,\ast \right) \]
Sei $\ast$ ein beliebiger arithmetischer Operator und $m \in \mathbb{R}, n \in \mathbb{N}$, dann sei $for$ definiert mit:
\[ \operatorname*{for}^n \left(m, \ast \right) = \operatorname*{\underbrace{m \ast \dotsb \ast m}}_{n} \]

\paragraph{For - Examples}

\[ \operatorname*{cfor}_{x=1}^{2}\left(\$+3,0\right) = 0+3+3 \]
\[ \operatorname*{for}_{x=1}^{2}\left(3,+\right) = 3+3 \]

\paragraph{Parameterized Operator}
\[ \operatorname*{for}^{n}\left(a,\operatorname*{\ast}^m\right) = \operatorname*{\underbrace{a \operatorname*{\ast}^m \dotsb \operatorname*{\ast}^m a}}_n = a\operatorname*{\ast}^{m+1}n\]
\[ a \operatorname*{\ast}^1 b = a + b \]
\end{document}
