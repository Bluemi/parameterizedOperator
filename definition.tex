\documentclass{article}
\usepackage{fullpage}
\usepackage{amsmath}
\usepackage{amssymb}

\begin{document}
\section{Parameterized Operator}
\paragraph{cfor - Definition \newline}

Sei $n \in \mathbb{N}$ die Anzahl der '$a$'s und sei $\ast$ ein beliebiger arithmetischer Operator und $a \in \mathbb{R}$, dann sei $cfor$ definiert mit:
\[ \operatorname*{cfor}^{n}\left(\$\ast a,y\right) = y\ast\underbrace{a\ast\dotsb\ast a}_{n} \]

\paragraph{for - Definition \newline}
Sei $\oslash\left(\ast\right)$ das neutrale Element der Operation $\ast$, dann sei $for$ definiert mit
\[ \operatorname*{for}^{n}\left(a,\ast \right) = \operatorname*{cfor}^{n}\left(\$\ast a,\oslash\left(\ast\right)\right)\]
\[ \operatorname*{for}^n \left(a, \ast \right) = \oslash\left(\ast\right) \ast \operatorname*{\underbrace{a \ast \dotsb \ast a}}_{n} \]

\paragraph{for - Examples}

\[ \operatorname*{cfor}^{2}\left(\$+3,0\right) = 0+3+3 \]
\[ = \operatorname*{for}^{2}\left(3,+\right) = 3+3 \]

\paragraph{Parameterized Operator}
\[ a \operatorname*{\ast}^1 b = a + b \]
\[ \operatorname*{for}^{n}\left(a,\operatorname*{\ast}^m\right) = \oslash\left(\operatorname*{\ast}^m \right) \ast \operatorname*{\underbrace{a \operatorname*{\ast}^m \dotsb \operatorname*{\ast}^m a}}_n = a\operatorname*{\ast}^{m+1}n\]

\end{document}
