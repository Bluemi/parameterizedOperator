\documentclass{article}
\usepackage{fullpage}
\usepackage{amsmath}
\usepackage{amssymb}

\begin{document}
\section{Rechengesetze f\"ur $\operatorname*{\ast}^{-1}$\newline}

\subsection{}
F\"ur $a, n \in \mathbb{N}$ gilt:
\[ \underbrace{a \operatorname*{\ast}^{-1} \dotsb \operatorname*{\ast}^{-1} a}_n = a\operatorname*{\ast}^0 n \]
Da $a\operatorname*{\ast}^0 n$ nur f\"ur $a=n$ definiert ist, gilt:
\[ \underbrace{a \operatorname*{\ast}^{-1} \dotsb \operatorname*{\ast}^{-1} a}_a = a\operatorname*{\ast}^0 a \]
Daraus resultiert, nach der Definition von $\operatorname*{\ast}^0$
\[ \underbrace{a \operatorname*{\ast}^{-1} \dotsb \operatorname*{\ast}^{-1} a}_a = a + 2 \]
\end{document}
