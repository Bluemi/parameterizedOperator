\documentclass{article}
\usepackage{fullpage}
\usepackage{amsmath}
\usepackage{amssymb}

\begin{document}
\section{Rechengesetze f\"ur $\operatorname*{\times}^0$\newline}

\subsection{}
F\"ur $a, n \in \mathbb{N}$ gilt:
\[ \underbrace{a \operatorname*{\times}^0 \dotsb \operatorname*{\times}^0 a}_n = a+n \]
Daher gilt f\"ur $a, n \in \mathbb{N}$:
\[ \underbrace{a \operatorname*{\times}^0 \dotsb \operatorname*{\times}^0 a}_n =
   a+n = \underbrace{n \operatorname*{\times}^0 \dotsb \operatorname*{\times}^0 n}_a \]
\subsection{}
Daraus folgt insbesondere:
\[ a + 0 = \underbrace{a\operatorname*{\times}^0 \dotsb \operatorname*{\times}^0 a}_0 \]
\[ a + 1 = \underbrace{a\operatorname*{\times}^0 \dotsb \operatorname*{\times}^0 a}_1 \]
\subsection{}
	Das Assoziativgesetz gilt nicht:
\[ 2 \operatorname*{\times}^0 2 \operatorname*{\times}^0 2 \operatorname*{\times}^0 2 \operatorname*{\times}^0 2 \operatorname*{\times}^0 2 \neq
   \left(2 \operatorname*{\times}^0 2 \operatorname*{\times}^0 2\right) \operatorname*{\times}^0 \left(2 \operatorname*{\times}^0 2 \operatorname*{\times}^0 2\right) \]
\[ 8 \neq 5 \operatorname*{\times}^0 5 \]
\[ 8 \neq 7 \]
	Zum Thema Assoziativit\"at gilt jedoch:
\[  \underbrace{\left(a+b\right)\operatorname*{\times}^0 \dotsb \operatorname*{\times}^0 \left(a+b\right)}_n =
    \underbrace{\underbrace{\left(a\operatorname*{\times}^0 \dotsb\operatorname*{\times}^0 a\right)}_b \operatorname*{\times}^0 \dotsb \operatorname*{\times}^0\underbrace{\left(a\operatorname*{\times}^0 \dotsb\operatorname*{\times}^0 a\right)}_b }_n \]
\[ = a + b + n = \underbrace{a\operatorname*{\times}^0 \dotsb \operatorname*{\times}^0 a}_{b+n} \]
\subsection{}
	Eine vereinfachte Form des Distributivgesetzes gilt:
\[ \left(a + 2\right) + c = \left(a\operatorname*{\times}^0a\right) + c = \left(a+c\right)\operatorname*{\times}^0\left(a+c\right) = 2 + \left(a + c\right) \]
\subsection{}
F\"ur $a, z, n \in \mathbb{N}$ gilt:
\[ a + n = \underbrace{a \operatorname*{\times}^0 \dotsb \operatorname*{\times}^0 a}_n =
   \underbrace{\left(a - z\right) \operatorname*{\times}^0 \dotsb \operatorname*{\times}^0 \left(a - z\right)}_{n+z} \]
\subsection{}
Sei $\operatorname*{\otimes}^0$ der Antioperator von $\operatorname*{\times}^0$:
\[ a = \left(a\operatorname*{\times}^0 b\right)\operatorname*{\otimes}^0 b \]
dann gilt:
\[ a = \left(a+2\right)\operatorname*{\otimes}^0 a \]
\end{document}
