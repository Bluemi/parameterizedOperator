\documentclass{article}
\usepackage{fullpage}
\usepackage{amsmath}
\usepackage{amssymb}

\begin{document}
\section{Rechengesetze f\"ur $\operatorname*{\ast}^0$\newline}

\subsection{}
F\"ur $a, n \in \mathbb{N}, n > 1$ gilt:
\[ \underbrace{a \operatorname*{\ast}^0 \dotsb \operatorname*{\ast}^0 a}_n = a+n \]
Daher gilt f\"ur $a, n \in \mathbb{N}, n > 1, a > 1$:
\[ \underbrace{a \operatorname*{\ast}^0 \dotsb \operatorname*{\ast}^0 a}_n =
   a+n = \underbrace{n \operatorname*{\ast}^0 \dotsb \operatorname*{\ast}^0 n}_a \]
\subsection{}
	Das Assoziativgesetz gilt nicht:
\[ 2 \operatorname*{\ast}^0 2 \operatorname*{\ast}^0 2 \operatorname*{\ast}^0 2 \operatorname*{\ast}^0 2 \operatorname*{\ast}^0 2 \neq
   \left(2 \operatorname*{\ast}^0 2 \operatorname*{\ast}^0 2\right) \operatorname*{\ast}^0 \left(2 \operatorname*{\ast}^0 2 \operatorname*{\ast}^0 2\right) \]
\[ 8 \neq 5 \operatorname*{\ast}^0 5 \]
\[ 8 \neq 7 \]
\subsection{}
	Eine vereinfachte Form des Distributivgesetzes gilt:
\[ \left(a + 2\right) + c = \left(a\operatorname*{\ast}^0a\right) + c = \left(a+c\right)\operatorname*{\ast}^0\left(a+c\right) = 2 + \left(a + c\right) \]
\subsection{}
F\"ur $a, z, n \in \mathbb{N}, n > 1$ und $n+z > 1$ gilt:
\[ a + n = \underbrace{a \operatorname*{\ast}^0 \dotsb \operatorname*{\ast}^0 a}_n =
   \underbrace{\left(a - z\right) \operatorname*{\ast}^0 \dotsb \operatorname*{\ast}^0 \left(a - z\right)}_{n+z} \]
\end{document}
