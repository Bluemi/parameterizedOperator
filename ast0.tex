\documentclass{article}
\usepackage{fullpage}
\usepackage{amsmath}
\usepackage{amssymb}

\begin{document}
\section{Rechengesetze f\"ur $\operatorname*{\ast}^0$\newline}

\subsection{}
F\"ur $a, n \in \mathbb{N}$ gilt:
\[ \underbrace{a \operatorname*{\ast}^0 \dotsb \operatorname*{\ast}^0 a}_n = a+n \]
\subsection{}
F\"ur $a, b, c, z \in \mathbb{N}$ gilt:
\[ a+b\operatorname*{\ast}^0c = \left(a+b\right)\operatorname*{\ast}^0 c \]
und das Distributivgesetz:
\[ a+\left(b\operatorname*{\ast}^0c\right) = a + b \operatorname*{\ast}^0 a + c \]
\[ \left( a_1 \operatorname*{\ast}^0 a_2 \operatorname*{\ast}^0 \dotsb \operatorname*{\ast}^0 a_n \right) + z =
         a_1 + z \operatorname*{\ast}^0 a_2 + z \operatorname*{\ast}^0 \dotsb \operatorname*{\ast}^0 a_n +z \]
Und das Kommutativgesetz
\[ a \operatorname*{\ast}^0 b = b \operatorname*{\ast}^0 a \]
Und das Assoziativgesetz
\[ \left( a \operatorname*{\ast}^0 b \right) \operatorname*{\ast}^0 c = a \operatorname*{\ast}^0 \left( b \operatorname*{\ast}^0 c \right) =
   a \operatorname*{\ast}^0 b \operatorname*{\ast}^0 c \]
\subsection{}
F\"ur $a, z, n \in \mathbb{N}, n > 1$ und $n+z > 1$ gilt:
\[ a + n = \underbrace{a \operatorname*{\ast}^0 \dotsb \operatorname*{\ast}^0 a}_n =
   \underbrace{\left(a - z\right) \operatorname*{\ast}^0 \dotsb \operatorname*{\ast}^0 \left(a - z\right)}_{n+z} =
   \underbrace{\left(a \operatorname*{\ast}^0 \dotsb \operatorname*{\ast}^0 a\right)}_{n+z} - z \]
\end{document}
