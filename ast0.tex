\documentclass{article}
\usepackage{fullpage}
\usepackage{amsmath}
\usepackage{amssymb}

\begin{document}
\section{Rechengesetze f\"ur $\operatorname*{\ast}^0$\newline}

\subsection{}
F\"ur $a, z, n \in \mathbb{N}, n > 1$ gilt:
\[ a + n = \underbrace{a \operatorname*{\ast}^0 \dotsb \operatorname*{\ast}^0 a}_n =
   \underbrace{\left(a - z\right) \operatorname*{\ast}^0 \dotsb \operatorname*{\ast}^0 \left(a - z\right)}_{n+z} =
   \underbrace{\left(a \operatorname*{\ast}^0 \dotsb \operatorname*{\ast}^0 a\right)}_{n+z} - z \]
\subsection{}
F\"ur $n \in \mathbb{N}$ gilt:
\[ n = \left(n-2\right)\operatorname*{\ast}^{0}\left(n-2\right) \]
\subsection{}
F\"ur $a, n \in \mathbb{N}$ gilt:
\[ \underbrace{a \operatorname*{\ast}^0 \dotsb \operatorname*{\ast}^0 a}_n = a+n \]
\subsection{}
\[ a+b\operatorname*{\ast}^0c = \left(a+b\right)\operatorname*{\ast}^0 c \]
\subsection{}
F\"ur $a, b, c \in \mathbb{N}$ gilt das Distributivgesetz:
\[ a+\left(b\operatorname*{\ast}^0c\right) = a + b \operatorname*{\ast}^0 a + c \]

\end{document}
